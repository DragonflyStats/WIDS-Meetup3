
\documentclass{beamer}

\usepackage{amsmath}
\usepackage{amssymb}
\usepackage{framed}
\usepackage{subfiles}

\begin{document}
	
%ESRItalk
\begin{frame}[fragile]
The American Community Survey distributes downloadable data about United States communities. 
Download the 2006 microdata survey about housing for the state of Idaho using \texttt{download.file()} from here: 

\begin{verbatim}
https://dl.dropbox.com/u/7710864/data/csv_hid/ss06hid.csv

or here:

https://spark-public.s3.amazonaws.com/dataanalysis/ss06hid.csv 
\end{verbatim}
and load the data into \texttt{R}. 
\end{frame}
%=========================================================================== %

% You will use this data for the next several questions. 
\begin{frame}[fragile]
\noindent \textbf{\textit{Code Book}}\\
The code book, describing the variable names is here: 

\begin{verbatim}
https://dl.dropbox.com/u/7710864/data/PUMSDataDict06.pdf

or here: 

https://spark-public.s3.amazonaws.com/dataanalysis/PUMSDataDict06.pdf
\end{verbatim}
\end{frame}
%=========================================================================== %
\begin{frame}[fragile]

How many housing units in this survey were worth more than \$1,000,000?
% **53**

%\begin{framed}
%	\begin{verbatim}
%	# Download 2006 microdata survey 
%	# re: housing for Idaho using download.file()
%	# setwd("~/DA")
%	download.file(
%	'https://spark-public.s3.amazonaws.com/dataanalysis/ss06hid.csv',
%	"ss06hid.csv", method="curl")
%	
%	# Download the code book:
%	# download.file(
%	'https://spark-public.s3.amazonaws.com/dataanalysis/PUMSDataDict06.pdf',
%	"PUMSDataDict06.pdf", method="curl")
%	\end{verbatim}
%\end{framed}

\end{frame}

%=========================================================================================== %

\begin{frame}[fragile]

\begin{framed}
	\begin{verbatim}	
	# load the data into R
	idahoData <- read.csv("ss06hid.csv", header=TRUE)
	
	# Is it just Idaho data?
	table(idahoData$ST)
	#Check the PDF - what does 16 mean?
	
	#any missing data?
	summary(idahoData$ST)
	
	# How many housing units are worth
	# more than $1,000,000?
	table(idahoData$TYPE,idahoData$VAL)
	\end{verbatim}
\end{framed}

\end{frame}

%=========================================================================================== %

\begin{frame}[fragile]

\begin{framed}
	\begin{verbatim}
	#from local files
	idahoData <- read.csv("daquiz2.csv", header=TRUE)
	
	\end{verbatim}
\end{framed}

\end{frame}

%=========================================================================================== %

\begin{frame}[fragile]
\frametitle{Question 4}

\begin{itemize}
	\item Use the data you loaded from Question 3. 
	\item Consider the variable FES. 
	\item Which of the "tidy data" principles does this variable violate?
\end{itemize}

%%READY
%\textbf{\textit{Revision}}\\
%What are the three characteristics of tidy data?
%
%\begin{itemize}
%	\item ``\textit{\textbf{Tidy data}}" by Hadley Wickham (RStudio)
%	\item Submission to Journal of Statistical Software
%	\item (http://vita.had.co.nz/papers/tidy-data.pdf)
%\end{itemize}
%Three Principles from Hadley Wickham's paper
%\begin{itemize}
%	\item[1.] Each variable forms a column, 
%	\item[2.] Each observation forms a row, 
%	\item[3.] Each table/file stores data about one kind of observation.
%\end{itemize}

\begin{framed} 
	\begin{verbatim}
	# let's look!
	unique(idahoData$FES)
	\end{verbatim}
\end{framed} 

\end{frame}
%-----------------------------------------------------------------%

%\subsection*{ Question 5 }
\begin{frame}


\textbf{Options}
\begin{itemize}
	\item[(i)]  Each tidy data table contains information about only one type of observation.\\
	(Not so)
	
	\item[(ii)]  Each variable in a tidy data set has been transformed to be interpretable.
	(No)
	
	\item[(iii)]  Tidy data has no missing values.
	
	\item[(iv)]  Tidy data has one variable per column.
\end{itemize}
\end{frame}
%-----------------------------------------------------------------%

%\subsection*{ Question 5 }
\begin{frame}[fragile]
Use the data you loaded from Question 3. 

\begin{itemize}
	\item How many households have 3 bedrooms and and 4 total rooms? 
	\item How many households have 2 bedrooms and 5 total rooms? 
	\item How many households have 2 bedrooms and 7 total rooms?
\end{itemize}
\begin{framed}
	\begin{verbatim}
	#USING TABLE
	#Rooms on Rows , Bedrooms on Columns
	#dnn adds dimension names
	
	table(idahoData$RMS,idahoData$BDS,dnn=list("RMS","BDS"))
	
	\end{verbatim}
\end{framed}

\end{frame}
%============================================================= %
\begin{frame}[fragile]
Another Way of Doing it
\begin{framed}
	\begin{verbatim}
	# How many households have 3 bedrooms and 4 total rooms?
	nrow(idahoData[!is.na(idahoData$BDS) & idahoData$BDS==3 &
	!is.na(idahoData$BDS) & idahoData$RMS==4,])
	# How many households have 2 bedrooms and 5 total rooms?
	nrow(idahoData[!is.na(idahoData$BDS) & idahoData$BDS==2 &
	!is.na(idahoData$BDS) & idahoData$RMS==5,])
	# How many households have 2 bedrooms and 7 total rooms?
	nrow(idahoData[!is.na(idahoData$BDS) & idahoData$BDS==2 &
	!is.na(idahoData$BDS) & idahoData$RMS==7,])
	
	\end{verbatim}
\end{framed}
% **148, 386, 49**
\end{frame}

%-----------------------------------------------------------------%

%\subsection*{Question 6}
\begin{frame}
\begin{itemize}
	\item Use the data from Question 3. 
	\item Create a logical vector that identifies the households on greater than 10 acres who sold more than \$10,000 worth of agriculture products. 
	\item Assign that logical vector to the variable `\texttt{agricultureLogical}`. 
	\item Apply the `\texttt{which()} function like this to identify the rows of the data frame where the logical vector is `TRUE`.
\end{itemize}

\end{frame}
%====================================================== %
\begin{frame}[fragile]
\begin{framed} 
	\begin{verbatim}
	# Like this (this wont run yet)
	which(agricultureLogical) 
	\end{verbatim}
\end{framed} 
\end{frame}
%====================================================== %
\begin{frame}[fragile]
What are the first 3 values that result?

\begin{framed}
\begin{verbatim}
	# Showing off a bit
	q6cols <- c("ACR", "AGS")
	which(names(idahoData) %in% q6cols)  
	
	# logical vector
	agricultureLogical <- idahoData$ACR==3 & idahoData$AGS==6
	
	# and:
	which(agricultureLogical) 
	\end{verbatim}
\end{framed} 

%**125, 238, 262**
\end{frame}
%====================================================== %
\begin{frame}
	
\frametitle{Question 7}

\begin{itemize}
	\item Use the data from Question 3. 
	\item Create a logical vector that identifies the households on greater than 10 acres who
	sold more than \$10,000 worth of agriculture products. 
	\item Assign that logical vector to the variable \texttt{agricultureLogical}. 
	\item Apply the \texttt{which()} function like this to identify the rows of the 
	data frame where the logical vector is TRUE and assign it to the variable indexes. 
\end{itemize}

\end{frame}
%====================================================== %
\begin{frame}[fragile]
	
\begin{framed} 
\begin{verbatim}
	indexes =  which(agricultureLogical) 
\end{verbatim}
\end{framed} 

If your data frame for the complete data is called \texttt{dataFrame} you can create a data frame 
with only the above subset with the command: 

\end{frame}
%====================================================== %
\begin{frame}[fragile]
	
\begin{framed} 
	\begin{verbatim}
	subsetDataFrame  = dataFrame[indexes,] 
	\end{verbatim}
\end{framed} 

\noindent Note that we are subsetting this way because the NA values in the variables 
will cause problems if you subset directly with the logical statement. 

\end{frame}

%====================================================== %
\begin{frame}[fragile]
	
\noindent How many households in the subsetDataFrame have a missing value for the mortgage status 
(MRGX) variable?

\begin{framed} 
	\begin{verbatim}
	indexes <- which(agricultureLogical)
	subsetIdahoData <- idahoData[indexes,]
	
	# And then:
	nrow(subsetIdahoData[is.na(subsetIdahoData$MRGX),])
	\end{verbatim}
\end{framed} 

\end{frame}


%====================================================== %
\begin{frame}[fragile]
	
\frametitle{Question 8}
\begin{itemize}
	\item Use the data from Question 3.
	\item Apply `\texttt{strsplit()}` to split all the names of the data frame on the characters "wgtp". 
	\item What is the value of the 123 element of the resulting list?
\end{itemize}

\begin{framed} 
\begin{verbatim}
	List <- strsplit(names(idahoData), "wgtp")
	List[123]
\end{verbatim}
\end{framed} 

%**"" "15"**
\end{frame}
%====================================================== %
\begin{frame}[fragile]
\frametitle{Question 9}

What are the 0\% and 100\% quantiles of the variable \texttt{YBL}? Is there anything wrong with these values?
\textit{ Hint: you may need to use the \texttt{na.rm} parameter.}

\begin{framed} 
	\begin{verbatim}
	quantile(idahoData$YBL, na.rm=TRUE)
	#  0%  25%  50%  75% 100% 
	#  -1    3    5    7   25 
	\end{verbatim}
\end{framed} 

\end{frame}

%====================================================== %
\begin{frame}[fragile]	
\frametitle{Question 10}

In addition to the data from Question 3, the American Community Survey also collects data about populations. 
Using `\texttt{download.file()}`, download the population record data from: 

\begin{verbatim}
https://dl.dropbox.com/u/7710864/data/csv_hid/ss06pid.csv 

or here:

https://spark-public.s3.amazonaws.com/dataanalysis/ss06pid.csv
\end{verbatim}

\end{frame}
%====================================================== %
\begin{frame}[fragile]
	
\begin{itemize}
	\item Load the data into \texttt{R}. Assign the housing data from Question 3 to a data frame `\texttt{housingData}` and the population data from above to a data frame `populationData`.
	
	\item Use the merge command to merge these data sets based only on the common identifier "SERIALNO". 
	
	\item What is the dimension of the resulting data set? 
\end{itemize}
%[OPTIONAL] For fun, you might look at the data and see what happened when they merged.

\end{frame}
%========================================================================================== %
\begin{frame}[fragile]
	
%	download.file(
%	'https://spark-public.s3.amazonaws.com/dataanalysis/ss06pid.csv',
%	'ss06pid.csv', method='curl')
%	
\textbf{Merging Data Sets}
\begin{framed} 
	\begin{verbatim}

	housingData <- read.csv("ss06hid.csv", header=TRUE)
	popuData <- read.csv("ss06pid.csv", header=TRUE)
	
	dim(merge(housingData, 
	popuData, by="SERIALNO", all=TRUE))
	\end{verbatim}
\end{framed} 

\end{frame}

\end{document}