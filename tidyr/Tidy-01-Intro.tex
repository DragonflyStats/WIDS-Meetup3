%% - https://cran.r-project.org/web/packages/tidyr/vignettes/tidy-data.html
%=========================================================%
Data tidying

It is often said that 80\% of data analysis is spent on the cleaning and preparing data. And it's not just a first step, but it must be repeated many over the course of analysis as new problems come to light or new data is collected. To get a handle on the problem, this paper focuses on a small, but important, aspect of data cleaning that I call data tidying: structuring datasets to facilitate analysis.

The principles of tidy data provide a standard way to organise data values within a dataset. A standard makes initial data cleaning easier because you don't need to start from scratch and reinvent the wheel every time. The tidy data standard has been designed to facilitate initial exploration and analysis of the data, and to simplify the development of data analysis tools that work well together. Current tools often require translation. You have to spend time munging the output from one tool so you can input it into another. Tidy datasets and tidy tools work hand in hand to make data analysis easier, allowing you to focus on the interesting domain problem, not on the uninteresting logistics of data.

%=========================================================%
Defining tidy data {#sec:defining}

Happy families are all alike; every unhappy family is unhappy in its own way — Leo Tolstoy

Like families, tidy datasets are all alike but every messy dataset is messy in its own way. Tidy datasets provide a standardized way to link the structure of a dataset (its physical layout) with its semantics (its meaning). In this section, I'll provide some standard vocabulary for describing the structure and semantics of a dataset, and then use those definitions to define tidy data.
%=========================================================%
Data structure

Most statistical datasets are data frames made up of rows and columns. The columns are almost always labeled and the rows are sometimes labeled. The following code provides some data about an imaginary experiment in a format commonly seen in the wild. The table has two columns and three rows, and both rows and columns are labeled.

preg <- read.csv("preg.csv", stringsAsFactors = FALSE)
preg
#>           name treatmenta treatmentb
#> 1   John Smith         NA         18
#> 2     Jane Doe          4          1
#> 3 Mary Johnson          6          7
There are many ways to structure the same underlying data. The following table shows the same data as above, but the rows and columns have been transposed.

read.csv("preg2.csv", stringsAsFactors = FALSE)
#>   treatment John.Smith Jane.Doe Mary.Johnson
#> 1         a         NA        4            6
#> 2         b         18        1            7
The data is the same, but the layout is different. Our vocabulary of rows and columns is simply not rich enough to describe why the two tables represent the same data. In addition to appearance, we need a way to describe the underlying semantics, or meaning, of the values displayed in table.
%=========================================================%
Data semantics

A dataset is a collection of values, usually either numbers (if quantitative) or strings (if qualitative). Values are organised in two ways. Every value belongs to a variable and an observation. A variable contains all values that measure the same underlying attribute (like height, temperature, duration) across units. An observation contains all values measured on the same unit (like a person, or a day, or a race) across attributes.

A tidy version of the pregnancy data looks like this: (you'll learn how the functions work a little later)

\begin{framed}
\begin{verbatim}
library(tidyr)
library(dplyr)
preg2 <- preg %>% 
  gather(treatment, n, treatmenta:treatmentb) %>%
  mutate(treatment = gsub("treatment", "", treatment)) %>%
  arrange(name, treatment)
preg2
#>           name treatment  n
#> 1     Jane Doe         a  4
#> 2     Jane Doe         b  1
#> 3   John Smith         a NA
#> 4   John Smith         b 18
#> 5 Mary Johnson         a  6
#> 6 Mary Johnson         b  7
\end{verbatim}
\end{framed}

This makes the values, variables and observations more clear. The dataset contains 18 values representing three variables and six observations. The variables are:

name, with three possible values (John, Mary, and Jane).

treatment, with two possible values (a and b).

n, with five or six values depending on how you think of the missing value (1, 4, 6, 7, 18, NA)

The experimental design tells us more about the structure of the observations. In this experiment, every combination of of person and treatment was measured, a completely crossed design. The experimental design also determines whether or not missing values can be safely dropped. In this experiment, the missing value represents an observation that should have been made, but wasn't, so it's important to keep it. Structural missing values, which represent measurements that can't be made (e.g., the count of pregnant males) can be safely removed.

For a given dataset, it's usually easy to figure out what are observations and what are variables, but it is surprisingly difficult to precisely define variables and observations in general. For example, if the columns in the pregnancy data were height and weight we would have been happy to call them variables. If the columns were height and width, it would be less clear cut, as we might think of height and width as values of a dimension variable. If the columns were home phone and work phone, we could treat these as two variables, but in a fraud detection environment we might want variables phone number and number type because the use of one phone number for multiple people might suggest fraud. A general rule of thumb is that it is easier to describe functional relationships between variables (e.g., z is a linear combination of x and y, density is the ratio of weight to volume) than between rows, and it is easier to make comparisons between groups of observations (e.g., average of group a vs. average of group b) than between groups of columns.

In a given analysis, there may be multiple levels of observation. For example, in a trial of new allergy medication we might have three observational types: demographic data collected from each person (age, sex, race), medical data collected from each person on each day (number of sneezes, redness of eyes), and meteorological data collected on each day (temperature, pollen count).

Variables may change over the course of analysis. Often the variables in the raw data are very fine grained, and may add extra modelling complexity for little explanatory gain. For example, many surveys ask variations on the same question to better get at an underlying trait. In early stages of analysis, variables correspond to questions. In later stages, you change focus to traits, computed by averaging together multiple questions. This considerably simplifies analysis because you don't need a hierarchical model, and you can often pretend that the data is continuous, not discrete.

%=====================================================%

\end{document}
