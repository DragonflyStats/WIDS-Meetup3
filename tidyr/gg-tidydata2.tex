\documentclass[TIDYMASTER.tex]{subfiles} 
\begin{document} 
%=============================================================================== %
%\begin{frame}[fragile]
%	\frametitle{Tidy Data with \texttt{R}}
%	\Large
%The next sections will show you how to transform untidy data sets into tidy data sets.
%
%
%\end{frame}
%=============================================================================== %
\begin{frame}[fragile]
\frametitle{Tidy Data with \texttt{R}}
\Large
% %2.2 spread() and gather()
\begin{itemize}
\item The \textbf{tidyr} package by Hadley Wickham is designed to help you tidy your data.
\item \textbf{tidyr} contains four functions that alter the layout of tabular data sets, while preserving the values and relationships contained in the data sets.
\item The two most important functions in tidyr are \texttt{gather()} and \texttt{spread()}. 
\item Each relies on the idea of a key value pair.
\end{itemize}

\end{frame}
%=============================================================================== %
\begin{frame}[fragile]
\frametitle{Tidy Data with \texttt{R}}
\Large
% %2.2.1 key value pairs
\begin{itemize}
\item A key value pair is a simple way to record information. 
\item A pair contains two parts: a \textbf{key} that explains what the information describes, and a \textbf{value} that contains the actual information. 
%\item So for example,
\begin{framed}
\begin{verbatim}
Password: 0123456789 
\end{verbatim}
\end{framed}
%would be a key value pair.
\item \textit{0123456789} is the \textbf{value}, and it is associated with the \textbf{key} \textit{Password}.
\end{itemize}

\end{frame}
%=============================================================================== %
\begin{frame}[fragile]
	\frametitle{Tidy Data with \texttt{R}}
	\Large
\begin{itemize}
%\item Data values form natural key value pairs. 
%
%\item The value is the value of the pair and the variable that the value describes is the key. 

\item You could decompose table1 into a group of key value pairs, but it would cease to be a useful data set because you no longer know which values belong to the same observation (next slides).
\item In tidy data, each cell will contain a value and each column name will contain a key, but this doesn’t need to be the case for untidy data.
\end{itemize}
\end{frame}
%=============================================================================== %
\begin{frame}[fragile]
	\frametitle{Tidy Data with \texttt{R}}
	\Large
\begin{verbatim}
Country: Afghanistan
Country: Brazil
Country: China
Year: 1999
Year: 2000
Year: 2001
\end{verbatim}
\end{frame}
%=============================================================================== %
\begin{frame}[fragile]
	\frametitle{Tidy Data with \texttt{R}}
	\Large
\begin{verbatim}
Population:   19987071
Population:   20595360
Population:  172006362
Population:  174504898
Population: 1272915272
Population: 1280428583
Cases:    745
Cases:   2666
Cases:  37737
Cases:  80488
Cases: 212258
Cases: 213766
\end{verbatim}	

\end{frame}
%=============================================================================== %
%\begin{frame}[fragile]
%	\frametitle{Tidy Data with \texttt{R}}
%	\Large
%\begin{itemize}
%\item Every cell in a table of data contains one half of a key value pair, as does every column name. 
%
%% Consider table2.
%\end{itemize}
%
%
%%table2
%\end{frame}
%=============================================================================== %
\begin{frame}[fragile]
	\frametitle{Tidy Data with \texttt{R}}
\begin{verbatim}
## Source: local data frame [12 x 4]
## 
##        country year        key      value
## 1  Afghanistan 1999      cases        745
## 2  Afghanistan 1999 population   19987071
## 3  Afghanistan 2000      cases       2666
## 4  Afghanistan 2000 population   20595360
## 5       Brazil 1999      cases      37737
## 6       Brazil 1999 population  172006362
## 7       Brazil 2000      cases      80488
## 8       Brazil 2000 population  174504898
## 9        China 1999      cases     212258
## 10       China 1999 population 1272915272
## 11       China 2000      cases     213766
## 12       China 2000 population 1280428583
\end{verbatim}
\end{frame}
%=============================================================================== %
\begin{frame}[fragile]
	\frametitle{Tidy Data with \texttt{R}}
	\Large
\textbf{spead()}
\begin{itemize}
\item	In table2, the key column contains only keys (and not just because the column is labelled key). 
\item Conveniently, the value column contains the values associated with those keys.
\item You can use the \texttt{spread()} function to tidy this layout.
\end{itemize}

\end{frame}
%=============================================================================== %
\begin{frame}[fragile]
\frametitle{Tidy Data with \texttt{R}}
\Large
% 2.2.2 spread()
\texttt{spread()}
\begin{itemize}
\item \texttt{spread()} turns a pair of key:value columns into a set of tidy columns. 
\item To use \texttt{spread()}, pass it the name of a data frame, then the name of the key column in the data frame, and then the name of the value column. 
\item Pass the column names as they are; do not use quotes.
\item To tidy table2, you would pass \texttt{spread()} the key column and then the value column.
\end{itemize}
\end{frame}
%=============================================================================== %
\begin{frame}[fragile]
	\frametitle{Tidy Data with \texttt{R}}
	\large
%table2
\begin{verbatim}
## Source: local data frame [12 x 4]
## 
##        country year        key      value
## 1  Afghanistan 1999      cases        745
## 2  Afghanistan 1999 population   19987071
## 3  Afghanistan 2000      cases       2666
## 4  Afghanistan 2000 population   20595360
.....
\end{verbatim}
\end{frame}
%=============================================================================== %
\begin{frame}[fragile]
	\frametitle{Tidy Data with \texttt{R}}
	\large
\begin{verbatim}
library(tidyr)
spread(table2, key, value)

## Source: local data frame [6 x 4]
## 
##       country year  cases population
## 1 Afghanistan 1999    745   19987071
## 2 Afghanistan 2000   2666   20595360
## 3      Brazil 1999  37737  172006362
## 4      Brazil 2000  80488  174504898
## 5       China 1999 212258 1272915272
## 6       China 2000 213766 1280428583
\end{verbatim}
\end{frame}
%=============================================================================== %
\begin{frame}[fragile]
	\frametitle{Tidy Data with \texttt{R}}
	\Large
\begin{itemize}
\item \texttt{spread()} returns a copy of your data set that has had the \textbf{key} and \textbf{value} columns removed. 
\item In their place, \texttt{spread()} adds a new column for each unique value of the key column (i.e. new columns: cases and populations).
\item These unique values will form the column names of the new columns. 
\item \texttt{spread()} distributes the cells of the former value column across the cells of the new columns and truncates any non-key, non-value columns in a way that prevents duplication.
\end{itemize}


\end{frame}
%=============================================================================== %
\begin{frame}[fragile]
\frametitle{Tidy Data with \texttt{R}}
\Large
\begin{itemize}
\item \texttt{spread()} distributes a pair of key:value columns into a field of cells. The unique values of the key column become the column names of the field of cells.

\item You can see that \texttt{spread()} maintains each of the relationships expressed in the original data set. The output contains the four original variables, country, year, population, and cases.

\item And the values of these variables are grouped according to the orginal observations, but now the layout of these relationships is tidy.
\end{itemize}

\end{frame}
%=============================================================================== %
\begin{frame}[fragile]
	\frametitle{Tidy Data with \texttt{R}}
	\Large
\texttt{spread()} takes three optional arguments in addition to data, key, and value:
\begin{itemize}
	\item \texttt{fill}
	\item \texttt{convert}
	\item \texttt{drop}
\end{itemize}
\end{frame}
%=============================================================================== %
\begin{frame}[fragile]
	\frametitle{Tidy Data with \texttt{R}}
	\Large
\texttt{fill}
\begin{itemize}
	\item  If the tidy structure creates combinations of variables that do not exist in the original data set, \texttt{spread()} will place an \texttt{NA} in the resulting cells.
	\item (\texttt{NA} is \texttt{R}’s missing value symbol). 
	\item You can change this behaviour by passing \texttt{fill} an alternative value to use.
\end{itemize}
\end{frame}
%=============================================================================== %
\begin{frame}[fragile]
	\frametitle{Tidy Data with \texttt{R}}
	\Large
\texttt{convert}
\begin{itemize}
	\item If a value column contains multiple types of data, its elements will be saved as a single type, usually character strings. 
	\item As a result, the new columns created by \texttt{spread()} will also contain character strings.
	\item  If you set \texttt{convert = TRUE}, \texttt{spread()} will run \texttt{type.convert()} on each new column, which will convert strings to \textit{doubles (numerics)}, \textit{integers}, \textit{logicals}, \textit{complexes}, or \textit{factors}.
	\end{itemize}
\end{frame}
%=============================================================================== %
\begin{frame}[fragile]
	\frametitle{Tidy Data with \texttt{R}}
	\Large
\vspace{-1cm}
	\texttt{drop} 
\begin{itemize}
\item The \texttt{drop} argument controls how \texttt{spread()} handles factors in the key column. 
\item If you set \texttt{drop = FALSE}, spread will keep factor levels that do not appear in the key column, filling in the missing combinations with the value of fill.
\end{itemize}
\end{frame}

\end{document}