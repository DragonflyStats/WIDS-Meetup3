\documentclass[TIDYMASTER.tex]{subfiles} 
\begin{document} 
	
%=============================================================================== %
\begin{frame}[fragile]
\frametitle{Tidy Data with \texttt{R}}
\Large
%2.3 
\vspace{-1cm}
\noindent \texttt{separate()} and \texttt{unite()}
\begin{itemize}
\item \texttt{spread()} and \texttt{gather()} help you reshape the layout of your data to place variables in columns and observations in rows.  \bigskip
\item \texttt{separate()} and \texttt{unite()} allow you split and combine cells to place a single, complete value in each cell.
\end{itemize}


\end{frame}
%=============================================================================== %
\begin{frame}[fragile]
	\frametitle{Tidy Data with \texttt{R}}
	\Large
% 2.3.1 
\noindent \texttt{ separate() }\\
\begin{itemize}
\item \texttt{separate()} turns a single character column into multiple columns by splitting the values of the column wherever a separator character appears.
%
%[SEPARATE DIAGRAM] 
\item So, for example, we can use \texttt{separate()} to tidy table3, which combines values of cases and population in the same column.
\end{itemize}
\end{frame}
%=============================================================================== %
\begin{frame}[fragile]
	\frametitle{Tidy Data with \texttt{R}}
	\large
	(\textbf{BEFORE)}
	\begin{framed}
		\begin{verbatim}
		# Data set three
		table3
		
		## Source: local data frame [6 x 3]
		## 
		##       country year              rate
		## 1 Afghanistan 1999      745/19987071
		## 2 Afghanistan 2000     2666/20595360
		## 3      Brazil 1999   37737/172006362
		## 4      Brazil 2000   80488/174504898
		## 5       China 1999 212258/1272915272
		## 6       China 2000 213766/1280428583
		\end{verbatim}
	\end{framed}
\end{frame}
%=============================================================================== %
\begin{frame}[fragile]
	\frametitle{Tidy Data with \texttt{R}}
	\Large
\begin{verbatim}
separate(table3, rate, 
    into = c("cases", "population"))

## Source: local data frame [6 x 4]
## 
##       country year  cases population
## 1 Afghanistan 1999    745   19987071
## 2 Afghanistan 2000   2666   20595360
## 3      Brazil 1999  37737  172006362
## 4      Brazil 2000  80488  174504898
## 5       China 1999 212258 1272915272
## 6       China 2000 213766 1280428583
\end{verbatim}
\end{frame}
%=============================================================================== %
\begin{frame}[fragile]
\frametitle{Tidy Data with \texttt{R}}
\Large
\begin{itemize}
\item To use \texttt{separate()} pass separate the name of a data frame to reshape and the name of a column to separate. 
\item Also give \texttt{separate()} an into argument, which should be a vector of character strings to use as new column names. 
\item \texttt{separate()} will return a copy of the data frame with the column removed. 
\item The previous values of the column will be split across several columns, one for each name in into.
\end{itemize}

\end{frame}
%=============================================================================== %
\begin{frame}[fragile]
\frametitle{Tidy Data with \texttt{R}}
\Large
\vspace{-0.4cm}
\noindent \textbf{Where to Separate?}
\begin{itemize}
\item By default, \texttt{separate()} will split values wherever a non-alphanumeric character appears.
\item Non-alphanumeric characters are characters that are neither a number nor a letter. 
\item For example, in the code above, \texttt{separate()} split the values of rate at the forward slash characters.
\end{itemize}

\end{frame}
%=============================================================================== %
\begin{frame}[fragile]
	\frametitle{Tidy Data with \texttt{R}}
	\Large
\noindent \textbf{Specifying a Character} \\
If you wish to use a specific character to separate a column, you can pass the character to the sep argument of \texttt{separate()}. 
%For example, we could rewrite the code above as

\begin{framed}
\begin{verbatim}
separate(table3, rate, 
    into = c("cases", "population"), 
    sep = "/")
\end{verbatim}
\end{framed}
% % See Appendix E to learn more about regular expressions in R.
\end{frame}
%=============================================================================== %
\begin{frame}[fragile]
\frametitle{Tidy Data with \texttt{R}}
\Large
\noindent \textbf{Multiple Separation}

\begin{itemize}
\item You can also pass an integer or vector of integers to \texttt{sep}. \texttt{separate()} will interpret the integers as positions to split at.

\item  Positive values start at 1 at the far-left of the strings; \item negative value start at -1 at the far-right of the strings. 
		
		\item The length of \texttt{sep} should be one less than the number of names in into. 
\end{itemize}
\end{frame}
%=============================================================================== %
\begin{frame}[fragile]
	\frametitle{Tidy Data with \texttt{R}}
	\Large

	\begin{itemize}
		

\item \textbf{Example:} You can use this arrangement to separate the last two digits of each year.
\end{itemize}

\end{frame}
%=============================================================================== %
\begin{frame}[fragile]
\frametitle{Tidy Data with \texttt{R}}
\large
\textbf{(Mid Columns : year into century and year)}
\begin{verbatim}
separate(table3, year, 
   into = c("century", "year"), sep = 2)

## Source: local data frame [6 x 4]
## 
##       country century year              rate
## 1 Afghanistan      19   99      745/19987071
## 2 Afghanistan      20   00     2666/20595360
## 3      Brazil      19   99   37737/172006362
## 4      Brazil      20   00   80488/174504898
## 5       China      19   99 212258/1272915272
## 6       China      20   00 213766/1280428583
\end{verbatim}


\end{frame}
%=============================================================================== %
%\begin{frame}[fragile]
%\frametitle{Tidy Data with \texttt{R}}
%\Large
%You can futher customize separate() with the remove, convert, and extra arguments:
%\begin{description}
%\item[remove] - Set \texttt{remove = FALSE} to retain the column of values that were separated in the final data frame.
%\item[convert] - By default, \texttt{separate()} will return new columns as character columns. Set convert = TRUE to convert new columns to double (numeric), integer, logical, complex, and factor columns with \texttt{type.convert().}
%\end{description}
%
%\end{frame}
%=============================================================================== %
%\begin{frame}[fragile]
%	\frametitle{Tidy Data with \texttt{R}}
%	\Large
%\texttt{extra} 
%\begin{itemize}
%
%
%\item extra controls what happens when the number of new values in a cell does not match the number of new columns in into.
%\item If extra = error (the default), separate() to return an error. 
%\item If extra = drop, separate() will drop new values and supply NAs as necessary to fill the new columns. 
%\item If extra = merge, separate() will split at most length(into) times.
%\end{itemize}
%
%\end{frame}

\end{document}