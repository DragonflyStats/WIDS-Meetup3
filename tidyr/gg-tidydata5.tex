\documentclass[TIDYMASTER.tex]{subfiles} 
\begin{document} 
	
%=============================================================================== %
\begin{frame}[fragile]
\frametitle{Tidy Data with \texttt{R}}
\Large
\vspace{-1cm}
\noindent \textbf{\texttt{unite()}}
\begin{itemize}
\item 
\texttt{unite()} does the opposite of \texttt{separate()}: it combines multiple columns into a single column.


\item We can use \texttt{unite()} to rejoin the century and year columns that we created in the last example. 
% That data is saved in the DSR package as table6.
\end{itemize}


\end{frame}
%=============================================================================== %
\begin{frame}[fragile]
\frametitle{Tidy Data with \texttt{R}}
\large
\begin{framed}
\begin{verbatim}
table6

## Source: local data frame [6 x 4]
## 
##       country century year              rate
## 1 Afghanistan      19   99      745/19987071
## 2 Afghanistan      20   00     2666/20595360
## 3      Brazil      19   99   37737/172006362
## 4      Brazil      20   00   80488/174504898
## 5       China      19   99 212258/1272915272
## 6       China      20   00 213766/1280428583
\end{verbatim}
\end{framed}

\end{frame}
%=============================================================================== %
\begin{frame}[fragile]
\frametitle{Tidy Data with \texttt{R}}{\large
\begin{framed}
\begin{verbatim}
unite(table6, "new", century, year, sep = "")

## Source: local data frame [6 x 3]
## 
##       country  new              rate
## 1 Afghanistan 1999      745/19987071
## 2 Afghanistan 2000     2666/20595360
## 3      Brazil 1999   37737/172006362
## 4      Brazil 2000   80488/174504898
## 5       China 1999 212258/1272915272
## 6       China 2000 213766/1280428583
\end{verbatim}
\end{framed}
}

\end{frame}
%=============================================================================== %
\begin{frame}[fragile]
\frametitle{Tidy Data with \texttt{R}}
\Large

\begin{itemize}
\item Give \texttt{unite()} the name of the data frame to reshape, the name of the new column to create (as a character string), and the names of the columns to unite. 
\bigskip
\item \texttt{unite()} will place an underscore (\_) between values from separate columns. 
\end{itemize}

\end{frame}
%=============================================================================== %
\begin{frame}[fragile]
	\frametitle{Tidy Data with \texttt{R}}
	\Large
	
	\begin{itemize}
\item If you would like to use a different separator, or no separator at all, pass the separator as a character string to \texttt{sep}.
\bigskip
\item \texttt{unite()} returns a copy of the data frame that includes the new column, but not the columns used to build the new column.  \bigskip
\item If you would like to retain these columns, add the argument \texttt{remove = FALSE}.
\end{itemize}

\end{frame}
%=============================================================================== %
%\begin{frame}[fragile]
%\frametitle{Tidy Data with \texttt{R}}
%\Large
%\begin{itemize}
%\item You can also use integers or the syntax of the dplyr::select to specify columns to unite in a more concise way. 
%\item We’ll learn about select in Section 3.1.
%\end{itemize}
%
%
%\end{frame}
\end{document}
