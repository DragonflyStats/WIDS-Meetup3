% - http://www.r-bloggers.com/new-package-dplyrr-utilities-for-comfortable-use-of-dplyr-with-databases/

1. Overview
dplyr is the most powerful package for data handling in R, and it has also the ability of working with databases(See Vignette). 
But the functionalities of dealing with databases in dplyr is developing yet.

Now, I’m trying to make dplyr with databases more comfortable by using some functions. 
For that purpose, I’ve created dplyrr package.

dplyrr has below functions:

load_tbls() : Easy to load table objects for all tables in a database.
cut() in mutate() : Easy to create a case statement by using the grammar like the base::cut().
count_if() and n_if() in summarise() : Shortcut to count rows that a condition is satisfied.
filter() : Improved filter() for tbl_sql which adds parentheses appropriately.
moving_mean() in mutate() : Compute moving average for PostgreSQL.
moving_max() in mutate() : Compute moving max for PostgreSQL.
moving_min() in mutate() : Compute moving min for PostgreSQL.
moving_sum() in mutate() : Compute moving sum for PostgreSQL.
first_value() in mutate() : Compute first value for PostgreSQL.
2. How to install
The source code for dplyrr package is available on GitHub at

https://github.com/hoxo-m/dplyrr
You can install the pakage from there.

install.packages("devtools") # if you have not installed "devtools" package
devtools::install_github("hoxo-m/dplyrr")
3. Common functions for all databases
For illustration, we use a database file: “my_db.sqlite3″. 
If you want to trace the codes below, you should first create the databese file.

library(dplyrr)
library(nycflights13)

db <- src_sqlite("my_db.sqlite3", create = TRUE)
copy_nycflights13(db)
3-1. load_tbls()
Usually, when we use a database with dplyr, we first create database object, and we can see the tables in the databese by show().

library(dplyrr)
# Create database object
db <- src_sqlite("my_db.sqlite3")
show(db)

## src:  sqlite 3.8.6 [my_db.sqlite3]
## tbls: airlines, airports, flights, planes, sqlite_stat1, weather
Next, we create table objects for pulling data in some tables in the database.

airlines_tbl <- tbl(db, "airlines")
airports_tbl <- tbl(db, "airports")
flights_tbl <- tbl(db, "flights")
planes_tbl <- tbl(db, "planes")
weather_tbl <- tbl(db, "weather")
Typing this code is really a bore!

If you want to create table objects for all tables in the database, you can use load_tbls().

load_tbls(db)

## Loading: airlines_tbl
## Loading: airports_tbl
## Loading: flights_tbl
## Loading: planes_tbl
## Loading: sqlite_stat1_tbl
## Loading: weather_tbl
Check the created table objects.

ls(pattern = "_tbl$")

## [1] "airlines_tbl"     "airports_tbl"     "flights_tbl"     
## [4] "planes_tbl"       "sqlite_stat1_tbl" "weather_tbl"
glimpse(airlines_tbl)

## Observations: 16
## Variables:
## $ carrier (chr) "9E", "AA", "AS", "B6", "DL", "EV", "F9", "FL", "HA", ...
## $ name    (chr) "Endeavor Air Inc.", "American Airlines Inc.", "Alaska...
3-2. cut() in mutate()
If you want to write case statement with like base::cut(), you can use cut() function in mutate().

For example, there is air_time column in the database.

db <- src_sqlite("my_db.sqlite3")
flights_tbl <- tbl(db, "flights")
q <- flights_tbl %>% select(air_time)
air_time <- q %>% collect
head(air_time, 3)

## Source: local data frame [3 x 1]
## 
##   air_time
## 1      227
## 2      227
## 3      160
If you want to group the air_time by break points c(0, 80, 120, 190, 900), you think you must write the next code.

q <- flights_tbl %>% 
  select(air_time) %>%
  mutate(air_time_cut = if(air_time > 0 && air_time <= 80) "(0,80]"
         else if(air_time > 80 && air_time <= 120) "(80,120]"
         else if(air_time > 120 && air_time <= 190) "(120,190]"
         else if(air_time > 190 && air_time <= 900) "(190,900]")
air_time_with_cut <- q %>% collect
head(air_time_with_cut, 3)

## Source: local data frame [3 x 2]
## 
##   air_time air_time_cut
## 1      227    (190,900]
## 2      227    (190,900]
## 3      160    (120,190]
When the break points increase, you are going to be tired to write more lines.

By using cut() function in mutate(), it can become easy.

q <- flights_tbl %>% 
  select(air_time) %>%
  mutate(air_time_cut = cut(air_time, breaks=c(0, 80, 120, 190, 900)))
air_time_with_cut <- q %>% collect
head(air_time_with_cut, 3)

## Source: local data frame [3 x 2]
## 
##   air_time air_time_cut
## 1      227    (190,900]
## 2      227    (190,900]
## 3      160    (120,190]
The cut() in mutate() has more arguments such as labels coming from base::cut().

cut(variable, breaks, labels, include.lowest, right, dig.lab)
For integer break points, specially you can indicate labels="-".

q <- flights_tbl %>% 
  select(air_time) %>%
  mutate(air_time_cut = cut(air_time, breaks=c(0, 80, 120, 190, 900), labels="-"))
air_time_with_cut <- q %>% collect
head(air_time_with_cut, 3)

## Source: local data frame [3 x 2]
## 
##   air_time air_time_cut
## 1      227      191-900
## 2      227      191-900
## 3      160      121-190
3-3. count_if() and n_if() in summarise()
When we want to count rows that a condition is satisfied, we write like this.

q <- flights_tbl %>% 
  select(air_time) %>%
  summarise(odd_airtime_rows = sum(if(air_time %% 2 == 1) 1L else 0L), 
            even_airtime_rows = sum(if(air_time %% 2 == 0) 1L else 0L), 
            total_rows=n())
q %>% collect

## Source: local data frame [1 x 3]
## 
##   odd_airtime_rows even_airtime_rows total_rows
## 1           164150            163196     336776
The count_if() and n_if() functions are a shortcut for it merely.

count_if(condition)
n_if(condition)
q <- flights_tbl %>% 
  select(air_time) %>%
  summarise(odd_airtime_rows = count_if(air_time %% 2 == 1), 
            even_airtime_rows = n_if(air_time %% 2 == 0), 
            total_rows=n())
q %>% collect

## Source: local data frame [1 x 3]
## 
##   odd_airtime_rows even_airtime_rows total_rows
## 1           164150            163196     336776
Both functions do exactly the same thing.

3-4. Improved filter()
If you use dplyr with databases in pure mind, you can encounter the unintended action like below.

library(dplyr)

db <- src_sqlite("my_db.sqlite3")
flights_tbl <- tbl(db, "flights")
q <- flights_tbl %>%
  select(month, air_time) %>%
  filter(month == 1) %>%
  filter(air_time > 200 || air_time < 100)
q$query

## <Query> SELECT "month" AS "month", "air_time" AS "air_time"
## FROM "flights"
## WHERE "month" = 1.0 AND "air_time" > 200.0 OR "air_time" < 100.0
## <SQLiteConnection>
Did you expect the WHERE clause to be that?

If you use dplyrr, it becomes natural by adding parentheses.

library(dplyrr)

db <- src_sqlite("my_db.sqlite3")
flights_tbl <- tbl(db, "flights")
q <- flights_tbl %>%
  select(month, air_time) %>%
  filter(month == 1) %>%
  filter(air_time > 200 || air_time < 100)
q$query

## <Query> SELECT "month" AS "month", "air_time" AS "air_time"
## FROM "flights"
## WHERE ("month" = 1.0) AND ("air_time" > 200.0 OR "air_time" < 100.0)
## <SQLiteConnection>
4. Functions for PostgreSQL
4-1. moving_**() in mutate()
dplyrr has four moving_**() functions that you can use in mutate().

moving_mean(variable, preceding, following)
moving_max(variable, preceding, following)
moving_min(variable, preceding, following)
moving_sum(variable, preceding, following)
When you want to set the same preceding and following, you can omit following.

For illustration, we use the test database that is PostgreSQL.

srcs <- temp_srcs("postgres")
df <- data.frame(x = 1:5)
tbls <- dplyr:::temp_load(srcs, list(df = df))
temp_tbl <- tbls$postgres$df
head(temp_tbl)

##   x
## 1 1
## 2 2
## 3 3
## 4 4
## 5 5
Compute moving average with 1 preceding and 1 following.

q <- temp_tbl %>%
  mutate(y = moving_mean(x, 1))
q %>% collect

## Source: local data frame [5 x 2]
## 
##   x   y
## 1 1 1.5
## 2 2 2.0
## 3 3 3.0
## 4 4 4.0
## 5 5 4.5
Comfirm query.

q$query

## <Query> SELECT "x", "y"
## FROM (SELECT "x", avg("x") OVER (ROWS BETWEEN 1 PRECEDING AND 1 FOLLOWING) AS "y"
## FROM "tlsqbjsuou") AS "_W1"
## <PostgreSQLConnection:(10316,0)>
Compute moving mean with 1 preceding and 2 following.

q <- temp_tbl %>%
  mutate(y = moving_mean(x, 1, 2))
q %>% collect

## Source: local data frame [5 x 2]
## 
##   x   y
## 1 1 2.0
## 2 2 2.5
## 3 3 3.5
## 4 4 4.0
## 5 5 4.5
Comfirm query.

q$query

## <Query> SELECT "x", "y"
## FROM (SELECT "x", avg("x") OVER (ROWS BETWEEN 1 PRECEDING AND 2 FOLLOWING) AS "y"
## FROM "tlsqbjsuou") AS "_W2"
## PostgreSQLConnection:(10316,0)>
Similary, you can use the other moving_**() functions.

4-2. first_value() in mutate()
dplyrr has first_value() function that you can use in mutate().

first_value(value, order_by)
When you want to set the same value and order_by, you can omit order_by.

For illustration, we use the test database that is PostgreSQL.

srcs <- temp_srcs("postgres")
df <- data.frame(class = c("A", "A", "B", "B", "C", "C"), x = 1:6, y = 6:1)
tbls <- dplyr:::temp_load(srcs, list(df=df))
temp_tbl <- tbls$postgres$df
head(temp_tbl)

##   class x y
## 1     A 1 6
## 2     A 2 5
## 3     B 3 4
## 4     B 4 3
## 5     C 5 2
## 6     C 6 1
Get the first values of x partitioned by class and ordered by x.

q <- temp_tbl %>%
  group_by(class) %>%
  mutate(z = first_value(x))
q %>% collect

## Source: local data frame [6 x 4]
## Groups: class
## 
##   class x y z
## 1     A 1 6 1
## 2     A 2 5 1
## 3     B 3 4 3
## 4     B 4 3 3
## 5     C 5 2 5
## 6     C 6 1 5
See query.

q$query

## <Query> SELECT "class", "x", "y", "z"
## FROM (SELECT "class", "x", "y", first_value("x") OVER (PARTITION BY "class" ORDER BY "x") AS "z"
## FROM "slrhxfdvrt") AS "_W3"
## <PostgreSQLConnection:(10316,0)>
Get the first values of x partitioned by class and ordered by y.

q <- temp_tbl %>%
  group_by(class) %>%
  mutate(z = first_value(x, y))
q %>% collect

## Source: local data frame [6 x 4]
## Groups: class
## 
##   class x y z
## 1     A 2 5 2
## 2     A 1 6 2
## 3     B 4 3 4
## 4     B 3 4 4
## 5     C 6 1 6
## 6     C 5 2 6
See query.

q$query

## <Query> SELECT "class", "x", "y", "z"
## FROM (SELECT "class", "x", "y", first_value("x") OVER (PARTITION BY "class" ORDER BY "y") AS "z"
## FROM "slrhxfdvrt") AS "_W4"
## <PostgreSQLConnection:(10316,0)>
Get the first values of x partitioned by class and ordered by descent of y.

q <- temp_tbl %>%
  group_by(class) %>%
  mutate(z = first_value(x, desc(y)))
q %>% collect

## Source: local data frame [6 x 4]
## Groups: class
## 
##   class x y z
## 1     A 1 6 1
## 2     A 2 5 1
## 3     B 3 4 3
## 4     B 4 3 3
## 5     C 5 2 5
## 6     C 6 1 5
See query.

q$query

## <Query> SELECT "class", "x", "y", "z"
## FROM (SELECT "class", "x", "y", first_value("x") OVER (PARTITION BY "class" ORDER BY "y" DESC) AS "z"
## FROM "slrhxfdvrt") AS "_W5"
## <PostgreSQLConnection:(10316,0)>

\end{frame}
%====================================================================%
\begin{frame}

5. Miscellaneous
update_dplyrr()
update_dplyrr() is a shortcut of

devtools::install_github("hoxo-m/dplyrr")
unload_dplyrr()
unload_dplyrr() is a shortcut of

detach("package:dplyrr", unload = TRUE)
detach("package:dplyr", unload = TRUE)
6. Bug reports
https://github.com/hoxo-m/dplyrr/issues
