
%---------------------------------------%
\subsection{\texttt{n_distinct}}

Efficiently count the number of unique values in a vector.


Package:  dplyr
Version:  0.1.1

%---------------------------------------%
\subsection{Description}
This is a faster and more concise equivalent of length(unique(x))


\begin{description}
\item[Usage:] n_distinct(x)

\item[Arguments:] x a vector of values
\end{description}
%---------------------------------------%
\subsection{Examples}

x <- sample(1:10, 1e5, rep = TRUE)
length(unique(x))
n_distinct(x)


In dplyr 0.3 this can be easily achieved using the distinct() method. Here is an example:


distinct_df = df %>% distinct(field1)
Then you can get a vector of the distinct values using: 
distinct_vector = distinct_df$field1

It can be cleaner to also select only the column you are interested in at the same time as you perform the distinct() call, like so:


distinct_df = df %>% distinct(field1) %>% select(field1)
distinct_vector = distinct_df$field1


%---------------------------------------%
\subsection{\texttt{n_distinct}}

Efficiently count the number of unique values in a vector.


Package:  dplyr
Version:  0.1.1

%---------------------------------------%
\subsection{Description}
This is a faster and more concise equivalent of length(unique(x))


\begin{description}
\item[Usage:] n_distinct(x)

\item[Arguments:] x a vector of values
\end{description}
%---------------------------------------%
\subsection{Examples}

x <- sample(1:10, 1e5, rep = TRUE)
length(unique(x))
n_distinct(x)

