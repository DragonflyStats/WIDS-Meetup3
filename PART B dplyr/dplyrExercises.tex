
%==============================================================%

\begin{framed}
\begin{verbatim}

df1 <- data.frame( key1 = c("A","B","C","D") , 
                   var1 = 1:4 , 
                   var2 = "Cork","Sligo","Clare"," Kerry " )
df2 <- data.frame( key1 = c("B","C","D","E") , 
                   var1 = 2:5 , 
                   var3 = c("Oscar","LouLou","Charlie","Sally") )

\end{verbatim}
\end{framed}

%===============================================================%

\subsection{joins in dplyr}

\begin{itemize}
\item \texttt{left_join} 
\item \texttt{right_join}
\item \texttt{join}
% \item \texttt{ }
\end{itemize}

%===============================================================%

% dplyr

\texttt{distinct()}  and \texttt{n_distinct()}

\begin{itemize}
\item \texttt{distinct()}  
\item \texttt{n\_distinct()}
\end{itemize}


%===============================================================%

\texttt{transmute()} is very similar to \texttt{mutate()}, 
but with only the newly created variables being kept.

We will use conventional \texttt{R} programming, but we will show how the $\%\;\>\;\%$ 
operator can be used to do the same operations.

We will concentrate on the variables
\begin{itemize}
\item \texttt{tz}  - Time Zone
\item \texttt{alt} - Altitude
\end{itemize}

%================================================================%
\subsection*{Exercises}
\begin{item}
\item Determine the summary for Airports in timezone 5 only
\item Find the highest airport in each timezone
\item Find the three highest airports in each timezone
\end{itemize}

\begin{framed}
\begin{verbatim}

ap1 <- group_by(airports,tz)
ap2 <- arrange(ap1,desc(alt))
ap3 <- slice(ap2,1:3)

\end{verbatim}
\end{framed}

Some timezones do not have three airports. 
If there are only two airports in a certain timezone, only those two will 
be printed out.

%-------------------------------------------------------------------%
